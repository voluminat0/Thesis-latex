%korte intro waarover thesis gaat

%op statische manier udsp verifiëren
%javascript (higher order pl) (prototype based inheritance)
%zorgen dat je kan uitdrukken in uitbreiding JS (domein specifiek) (RPE als library).

%er zijn al anderen die geprobeerd hebben, 
%waarom RPE: security poli’s om af te checken bestaan uit control flow en dataflow constraints (beperkingen waarin instructies gevaluteerd worden)
%zegen ok we hebben met JIPDA ASG (precies opeenvolging van staten), leent zich om cf-properties af te checken. Checken is niet specificeren (RPE zeer geschikt).

%stel je wilt udsp afchecken (als user), dan moeten we wat voor soorten characteristics aanbieden,
%cfg is een, dataflow is ander, eigenschappen die JIPDA berekent uittrekbaar maken (store characteristics). Domein specifieke uitbreiding aan JS.

%vertrokken van JS en opzoek naar manier omdool om gebruikers zelf policies definiëren en ik kan statisch at compile time afchecken
%ingredients
%cf en df berekenen
%uitdrukken door gebruikers
%eens uitgedrukt matchen


\section{Objectives and contributions}
% Objectives -> finding bugs/vulnerabilities is hard, any help is useful
% Investigate whether RPE are geschikt voor security policies
% Taal
% Tool
\section{Overview}

%2.1 static analyse + abstract interpretation to be precise
%2.2 Generic vulnerabilities
%2.3 Application-specific vulnerabilities

%3.1 architectuur
%3.2 flow graphs for JS
%3.3 DSL’s (internal vs external vs global vs dal)
%3.4 Designing internal DSL + patterns + constraints

%4.1 Taal
%4.2 Type queries
%4.3 Defining predicates and policies +recursion

%5 implementation + dieper ingaan op implementatiedetails vooral van matching engine

%6 Evaluatie van our approach door policies uit verschillende papers na te bootsen en kijken of we kunnen uitdrukken

%7 Conclusion en future work

%Wat we bespreken in de thesis