In this chapter we validate and evaluate the expressiveness of the JS-QL query language by expressing some existing security policies, described in other related work, in our own query language. We will then compare these policies in terms of expressiveness and flexibility. %TODO: Vergelijken op welke manier
The concept and approach for creating a new, domain-specific language for security policies is explained in chapter \ref{ch:QueryLanguage}. Chapter \ref{ch:QueryEngine} discusses the underlying query engine and how it works together with the query language to process the application-specific policies. In chapter \ref{ch:Implementation} we explain how our approach was instantiated.

We start this chapter by expressing 9 security policies distilled from 3 papers in sections \ref{sec:ValidationPidginQL}, \ref{sec:ValidationGK} and \ref{sec:ValidationConscript} respectively. Every JS-QL policy will be evaluated by comparing how well it matches the policy expressed in the original paper. Finally, in section \ref{sec:ValidationEvaluation}, we evaluate the query framework by specifying its advantages, disadvantages and limitations. We will also briefly compare the query languages presented in this chapter in terms of expressiveness, verbosity and conciseness (\textit{LOC}).
%TODO: wat gaan we nog comparen?

\section{The PidginQL language}
\label{sec:ValidationPidginQL}

In this chapter we attempt to express 3 policies originally presented in \cite{PidginQLTechReport}.

\subsection{}
\subsection{}
\subsection{}

\section{the GateKeeper language}
\label{sec:ValidationGK}

\section{The Conscript language}
\label{sec:ValidationConscript}

\section{Evaluation}
\label{sec:ValidationEvaluation}
